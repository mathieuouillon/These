\documentclass[a4paper,11pt]{StyleThese}
\include{formatAndDefs}

\begin{document}
\dominitoc
\tableofcontents

\chapter{Filtre de Kalman}
\minitoc

\section{Filtre de Kalman}

Le Filtre de Kalman (KF) et son extension au temps continu appelé Filtre de Kalman-Bucy permettent de résoudre de façon optimale le problème du filtrage linéaire, quand les bruits du système sont additifs et gaussiens. Depuis 1960,
le filtre de Kalman a été intensivement utilisé dans des applications diverses tel que l'aéronautique, la balistique etc., et a donné lieu à de nombreuses recherches. 

L'algorithme de filtrage assume un systeme dynamique linéaire discret dans le temps dont l’évolution dynamique est modélisée à l’aide de l’équation d’état :

\begin{equation}
  x_k = F_kx_{k-1} + B_k u_{k-1} + w_k
  \label{equation d'etat}
\end{equation}


où $k \geq 0$ représente les instants successifs du temps, $F_k$ est la matrice d'état (dimension $(n\times x)$), $x_k$ est le vecteur des variables d'état (dimension $(n\times 1)$), $B_k$ est la matrice d'entrée (dimension $(n\times l)$), $u_k$ est le vecteur de contrôle (dimension $(l\times 1)$) et $w_k$ est le bruit du système (dimension $(n\times 1)$) (moyenne nulle et distribution gaussienne).
\\

Le filtre assume aussi la disponibilité d'un ensembles de mesures qui sont aussi une fonction linéaire des variables d'état :
\begin{equation}
  z_k = H_kx_k + v_k
  \label{equation mesure}
\end{equation}

où $H_k$ est la matrice de mesure et $v_k$ est le bruit de la mesure non correller avec $w_k$ (moyenne nulle et distribution gaussienne).

Le filtre de Kalman est un algorithme récursif qui exploite la connaissance du covariance attendue des variables d'état pour corriger l'estimation a priori des deux l'état du système et la covariance elle-même. 
L'algorithme peut être divisé en deux étapes (dans les équations suivantes, l'indice $_{k|k-1}$ après une variable indique sa valeur estimée avant la mise à jour de la mesure, tandis que l'indice $_{k|k}$ ou $_k$ indique l'estimation après la mise a jour).

\textbf{Etape de prédiction :} a cette étape, le vecteur d'état est prédit grace au modèle du système : 
\begin{equation}
  x_{k|k-1} = F_{k-1} x_{k-1} + B_{k-1}u_{k-1}
\end{equation}

La matrice de covariance au temps $k$ est aussi prédit : 
\begin{equation}
  P_{k|k-1} = F_{k-1}P_{k-1}F^T_{k-1} + Q_{k-1} 
\end{equation}
où $Q_k$ est la matrice de covariance du bruit du système $w_k$ dans l'équation \ref{equation d'etat}


\textbf{Etape de mise a jour :} a cette étape, la prediction a priori est mise à jour en fonction des mesures : 
\begin{equation}
  x_{k} = x_{k|k-1} + K_k(z_k - H_kx_{k|k-1})
  \label{mise a jour vecteur}
\end{equation}

et la matrice de covariance est aussi mise a jour : 
\begin{equation}
  P_k = (I-K_kH_k)P_{k|k-1}
  \label{mise a jour covariance}
\end{equation}
La matrice $K_k$ dans les équations \ref{mise a jour vecteur} et \ref{mise a jour covariance} est la matrice de gain de Kalman optimal au temps $k$ : 
\begin{equation}
  K_k = P_{k|k-1}H_k^T(H_kP_{k|k-1}H_k^T + R_k)^{-1}
\end{equation}
où $R_k$ est la matrice de covariance du bruit de mesure $v_k$ dans l'équation \ref{equation mesure}.
\section{Filtre de Kalman étendu}
L'une des principales limitations du filtre de Kalman est le fait qu'il nécessite à la fois
le système dynamique et les fonctions de mesure doivent être linéaires par rapport à la
variables d'état. Dans de nombreuses applications pratiques, ces exigences ne sont pas satisfaites.
Le filtre de Kalman étendu est une version modifiée du filtre de Kalman standard qui
peut être appliqué lorsque le système et/ou le modèle de mesure sont non linéaires.

L'équation \ref{equation d'etat non lineaire} est l'equation non linéaire qui décrit le modèle du système :
\begin{equation}
  x_k = f_k(x_{k-1}) + w_k
  \label{equation d'etat non lineaire}
\end{equation}
L'équation du modèle de mesure non linéaire est :
\begin{equation}
  z_k = h_k(x_k) + v_k
\end{equation}

\textbf{Étape de prediction :} 
\begin{equation}
  x_{k|k-1} = f_k(x_{k-1})
\end{equation}
\begin{equation}
  P_{k|k-1} = F_{k-1}P_{k-1}F^T_{k-1} + Q_{k-1} 
\end{equation}

\textbf{Etape de mise a jour :}
\begin{equation}
  x_{k} = x_{k|k-1} + K_k(z_k - h_k(x_{k|k-1}))
\end{equation}
\begin{equation}
  P_k = (I-K_kH_k)P_{k|k-1}
\end{equation}
\begin{equation}
  K_k = P_{k|k-1}H_k^T(H_kP_{k|k-1}H_k^T + R_k)^{-1}
\end{equation}
où $F_k$ et $H_k$ représentent les matrices jacobiennes de $f_k$ et $h_k$. Soit $F_k = \frac{\partial f}{\partial x_k}$ et $h_k = \frac{\partial h}{\partial x_k}$.
\section{Application du EKF au tracking dans BONuS}

\end{document}
